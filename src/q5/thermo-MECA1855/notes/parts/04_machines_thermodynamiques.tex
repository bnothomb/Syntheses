\section{Questions sur le chapitre ``Machines thermodynamiques et pertes de charges''}
\subsection{Pour les dissipations régulières, on a à chaque fois $W_f$ proportionnel à $c^x$. Donnez $x$ et justifiez pour différents régimes : laminaire, turbulent lisse et turbulent rugueux.}
Les dissipations régulières s'expriment ainsi :
\begin{equation} W_f = \lambda\frac{L}{D_h}\frac{c^2}{2} \end{equation}
Avec $\lambda(\text{Re}, \epsilon/D_h)$ et $\text{Re} = \frac{\rho cD_h}{\mu}$.
Dans le cas d'un régime laminaire, il n'y a pas de mélanges de vitesse transversaux. $\lambda = \frac{64}{\text{Re}}$, on a donc $W_f \sim c$. Dans le cas d'un régime turbulent, des fluctuations rapides de vitesse locale entraîne des échanges de quantité de mouvement entre les particules du fluide. Deux cas sont à prendre en compte :
\begin{itemize}
	\item Turbulent lisse : $\lambda = \lambda(\text{Re}^{-0.25}$, $W_f \sim c^{1.75}$;
	\item Turbulent rugueux : $\lambda = \lambda(\epsilon/D_h)$, $W_f \sim c^2$.
\end{itemize}
